The cKDTree module is rewritten in C++ with templated classes.

The support to a box of periodic boundary condition is added. A periodic
condition is typically used in computer simulations of physical processes.

\subsubsection{k-NN}

% TODO: please check / improve literature citations used here
% what about the time complexity of the kNN implementation?
In 2015, we enhanced \texttt{cKDTree.query} with a $k$ nearest neighbors search
parameter. This may be an efficient operation\cite{Sproull:1991:RNS:3118219.3118331} 
because it generates and returns a data structure that may be only a fraction 
of the size of the full neighbor list. A C++ implementation is provided that releases
the Python global interpreter lock for the neighbor search process, uses
a memory pool to allocate and automatically reclaim \texttt{structs}, and
heapsort to sort the $k$ nearest neighbors distances, even in the case
of periodic boundary conditions. When $k$ is a non-contiguous list of nearest
neighbors to account for (i.e., \texttt{[1, 3, 4]}), the intervening distances
between the nearest neighbor ($k = 1$) and the farthest neighbor requested
are all interrogated, but only those neighbors that are requested are retained
in memory. Thus, the max heap size for this algorithm is conceptually determined
as \texttt{np.arange(max(k))}.

\subsubsection{sparse distance matrix}

generating approximated sparse distance matrix

\subsubsection{Pair-Counting}

The cKDTree module implements the dual tree counting algorithm in (cite).

An improvement to the pair counting algorithm is added to improve the scale
with the number of bins.

\textbf{Add a Figure to show the scaling, before and after.}

The cKDTree can now be augmented by weights. Weighted paircount is essential
in many scientific applications, e.g. computing correlation function of galaxies.

\textbf{perhaps give an example or some formula.}

(cite / mention faster implementions of paircounting algorithms / treecorr, corrfunc)

The main purpose of the paircounting algorithm in cKDTree is to provide a readily
available tool. The KDTree based algorithm is not necessarily the fastest algorithm
depending on the application. E.g. for low number densities a chaining mesh based algorithm
can be easily many times faster than a tree based algorithm.

