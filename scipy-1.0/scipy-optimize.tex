\newcommand{\RR}{\ensuremath{\mathbb{R}}}
The \texttt{scipy.optimize} subpackage provides functions for the numerical
solution of several classes of root finding and optimization problems.
Here we highlight recent additions through SciPy 1.0.

%\subsubsection{Root Finding}
%The general ``root finding'' problem is to find a root $\mathbf{x} \in \RR^m$ of $\mathbf{f}: \RR^m \rightarrow \RR^m$, that is, to solve
%\begin{equation}
%\mathbf{f}(\mathbf{x}) = \mathbf{0}
%\end{equation}
%for a solution $\mathbf{x}$.\footnote{Equivalently the problem is to simultaneously find the roots $x_i \in \RR$ of several scalar functions $f_i : \RR \rightarrow \RR$, that is, to solve $f_i(x_0, x_1, \dots, x_{m-1}) = 0$ for $x_i$, $i \in \{0, 1, \dots {m-1}\}$.} The function \texttt{scipy.optimize.root} provides a common interface to several algorithms for solving problems of this type. For the special case\footnote{that is, to solve a single scalar equation $f(x) = 0$ for a single scalar variable $x$} $m = 1$, any one of several specialized functions \texttt{brentq}, \texttt{brenth}, \texttt{ridder}, \texttt{bisect}, or \texttt{newton} may provide improved performance or accuracy. (Have there been any recent improvements? Do we want to summarize the methods as @antonior92 has done for $minimize$? Do we have to explain that these methods only provide \emph{one} solution, and that they are iterative based on a user-provided guess? Do we have to explain the notion of tolerance? Is this a good template for the beginning of the following subsections?)


\subsubsection*{Linear Optimization}

A new interior-point optimizer for continuous linear programming problems, \texttt{linprog} with \texttt{method='interior-point'}, was released with SciPy 1.0. Implementing the core algorithm of the commercial solver MOSEK \cite{andersen2000mosek}, it solves all of the 90+ NETLIB LP benchmark problems \cite{netlib} tested. Unlike some interior point methods, this homogeneous self-dual formulation provides certificates of infeasibility or unboundedness as appropriate. 

A presolve routine \cite{andersen1995presolving} solves trivial problems and otherwise performs problem simplifications, such as bound tightening and removal of fixed variables, and one of several routines for eliminating redundant equality constraints is automatically chosen to reduce the chance of numerical difficulties caused by singular matrices. Although the main solver implementation is pure Python, end-to-end sparse matrix support and heavy use of SciPy's compiled linear system solvers---often for the same system with multiple right hand sides due to the predictor-corrector approach---provide speed sufficient for problems with tens of thousands of variables and constraints.

Compared to the previously implemented simplex method, the new interior-point method is faster for all but the smallest problems, and is suitable for solving medium- and large-sized problems on which the existing simplex implementation fails. However, the interior point method typically returns a solution near the center of an optimal face, yet basic solutions are often preferred for sensitivity analysis and for use in mixed integer programming algorithms. This motivates the need for a crossover routine or a new implementation of the simplex method for sparse problems in a future release, either of which would require an improved sparse linear system solver with efficient support for rank-one updates

\subsubsection*{Nonlinear Optimization}
\paragraph{Local Minimization}
The \texttt{minimize} function provides a unified interface for finding local minima of nonlinear optimization problems. Four new methods for unconstrained optimization were added to \texttt{minimize} in recent versions of SciPy: \texttt{dogleg}, \texttt{trust-ncg}, \texttt{trust-exact}, and \texttt{trust-krylov}. All are trust-region methods that build a local model of the objective function based on first and second derivative information, approximate the best point within a local ``trust region'', and iterate until a local minimum of the original objective function is reached, but each has unique characteristics that make it appropriate for certain types of problems. For instance, \texttt{trust-exact} achieves fast convergence by solving the trust-region subproblem almost exactly, but it requires the second derivative Hessian matrix to be stored and factored every iteration, which may preclude the solution of large problems ($\geq 1000$ variables). On the other hand, \texttt{trust-ncg} and \texttt{trust-krylov} are well-suited to large-scale optimization problems because they do not need to store and factor the Hessian explicitly, instead using second derivative information in a faster, approximate way. A detailed comparison of the characteristics of all \texttt{minimize} methods is presented in Table~\ref{tab:minimize-si}.

\paragraph{Global Minimization}
As \texttt{minimize} may return any local minimum, some problems require the use of a global optimization routine. The new \texttt{scipy.optimize.differential\textunderscore evolution} function \cite{Wormington1999,Storn1997} is a stochastic global optimizer that works by evolving a population of candidate solutions. In each iteration, trial candidates are generated by combination of candidates from the existing population. If the trial candidates represent an improvement, then the population is updated. Most recently, the SciPy benchmark suite gained a comprehensive set of 196 global optimization problems for tracking the performance of existing solvers over time and for evaluating whether the performance of new solvers merits their inclusion in the package.

