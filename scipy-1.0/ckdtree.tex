The \texttt{cKDTree} module was rewritten in C++ with templated classes, and support for
periodic boundary conditions was added. A periodic boundary condition is typically 
used in computer simulations of physical processes.

In 2013, the time complexity of the $k$-nearest neighbor search from
\texttt{cKDTree.query} was approximately loglinear \cite{knn-jake},
consistent with its formal description \cite{kdtree-search-algo}.
Since then, we've enhanced \texttt{cKDTree.query} by reimplementing it in
C++, removing memory leaks, and allowing release of the global interpreter lock (GIL) so that
multiple threads may be used\cite{gh-4374}. This generally improved
performance on any given problem (Figure~\ref{fig:asvbench}) while
preserving the asymptotic complexity (Figure~\ref{fig:knn-complexity}).

In 2015, SciPy added the \texttt{sparse\_distance\_matrix} routine for generating 
approximate sparse distance matrices between \texttt{KDTree} objects by ignoring 
all distances that exceed a user-provided value. Also, this routine is not 
limited to the conventional L2 (Euclidean) norm but supports any Minkowski 
p-norm between 1 and infinity. By default, the returned data structure is a 
Dictionary Of Keys based sparse matrix (DOK), which is very efficient for matrix 
construction. This hashing approach to sparse matrix assembly can be 7 times 
faster than constructing with Compressed Sparse Row format (CSR) 
\cite{10.1007/978-3-540-75755-9_107} and the C++ level sparse matrix construction 
releases the Python GIL for increased performance. Once the matrix is constructed, 
distance value retrieval has an amortized constant time complexity 
\cite{Cormen:2001:IA:580470}, and the DOK structure can be efficiently converted 
to a CSR, Compressed Sparse Column (CSC) or COOrdinate (COO) matrix to allow for 
speedy arithmetic operations.

In 2015 the \texttt{cKDTree} dual tree counting algorithm\cite{Moore2000ar}
was enhanced to support weights\cite{ckdtree-weights}, which are
essential in many scientific applications, e.g. computing correlation
functions of galaxies\cite{0004-637X-750-1-38}.
