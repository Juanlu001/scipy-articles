
Traditionally, SciPy relied heavily on the venerable \texttt{FITPACK}
Fortran library by P.~Dierckx \cite{Dierckx:1993:CSF:151103, FITPACK} for
univariate interpolation and approximation of data. FITPACK provides a
collection of routines for calculating smoothing splines and automatic knot
selection. One issue with FITPACK and its Python wrappers, \code{UnivariateSpline} and \code{splrep}/\code{splev} combo, is the monolithic design: specific
algorithms for construction of interpolating and smoothing splines and knot
selection are tightly coupled at the Fortran level. The library offers very
limited degree of user control over the details of the spline fitting it performs. 
%
An alternative set of spline interpolation routines was implemented in the
(now deprecated) \code{splmake}/\code{spleval} combo, and used as a backend of
the \code{interp1d} routine. To further complicate matters, the spline format
used by \code{splmake}/\code{spleval} was incompatible with
\code{splrep}/\code{splev}.
%
Routines for manipulating piecewise polynomials and b-splines were either not available or lacking in ease of use and/or performance characteristics.

Implementing a new, modular design of polynomial interpolators was spread over
several releases. The goals of this effort were to have a set of basic objects
representing piecewise polynomials, to implement a collection of algorithms
for constructing various interpolators, and to provide users with building
blocks for constructing additional interpolators.

At the lowest level of the new design are classes which represent univariate
piecewise polynomials: \code{PPoly}, \code{BPoly} and \code{BSpline}, which allow
efficient vectorized evaluations, differentiation, integration and root-finding.
\code{PPoly} represents piecewise polynomials in the power basis in terms of
breakpoints and coefficients at each interval. \code{BPoly} is similar, and
represents piecewise polynomials in the Bernstein basis (which is suitable
for e.g., constructing Bezier curves).

In the next layer, these polynomial classes are used for constructing several common ways of interpolating data: \code{CubicSpline} constructs a twice differentiable piecewise cubic function, \code{Akima1DInterpolator} and \code{PCHIPInterpolator} implement two classic prescriptions for constructing a $C^1$ continuous monotone shape-preserving interpolator. \cite{FritschCarlson1980, Akima1970}
%
Interpolation with known or estimated derivatives, i.e., cubic or higher-order Hermite splines, can be constructed by either subclassing \code{PPoly} or \code{BPoly}, or using the factory method \code{BPoly.from\_derivatives}.

For working with b-splines, there are several options. The basic object is \code{BSpline}, which represents spline curves, i.e., linear combinations of b-spline basis elements.\cite{deBoor1978}. The spline
representation used by \code{BSpline} is directly compatible with FITPACK. We
recommend that users work with \code{BSpline} objects instead of directly
manipulating \code{tck} tuples of knots, coefficients and the spline degree.
%
Using \code{UnivariateSpline} in new code is not recommended. Given data points,
the helper function \code{make\_interp\_spline} constructs a \code{BSpline}
instance, representing a spline which interpolates the data. Interpolated values
at intermediate points can be computed either via its \code{\_\_call\_\_} method,
or by calling \code{splev(xnew, BSpline\_object)} --- the \code{splev} routine
accepts both \code{tck} tuples and \code{Bspline} instances.

Fitting \emph{smoothing} splines to data can currently only be done with FITPACK. Building alternative algorithms is certainly possible, but is still to be implemented.

