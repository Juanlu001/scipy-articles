\documentclass[fleqn,10pt]{wlscirep}
\title{SciPy - some descriptive title here}

\author[1]{Pauli Virtanen}
\author[2,*]{Ralf Gommers}
\author[3]{TODO}
\affil[1]{Affiliation, department, city, postcode, country}
\affil[2]{Affiliation, department, city, postcode, country}
\affil[2]{Affiliation, department, city, postcode, country}

\affil[*]{ralf.gommers@gmail.com}

\keywords{Scientific computing, Python, Mathematics}

\begin{abstract}
TODO. Abstract must be under 200 words and not include subheadings or citations.
\end{abstract}
\begin{document}

\flushbottom
\maketitle
\thispagestyle{empty}

\section*{Introduction}

\textit{The Introduction section expands on the background of the work (some overlap with the Abstract is acceptable). The introduction should not include subheadings.}

\textbf{History}

\textbf{Project goals and scope}

\textbf{Current status (maturity, users)}


\section*{Architecture and implementation choices}

\subsection*{Submodule organisation}

\subsection*{Common infrastructure}

\subsection*{Language preferences}

\subsection*{API and ABI evolution}


\section*{Key technical improvements}

Here we describe key technical improvements made in the last three years.

\subsection*{Data structures}
\textbf{cKDTree}
\textbf{Sparse matrices}

\subsection*{Unified bindings to compiled code}
LowLevelCallable

\subsection*{Cython bindings for BLAS, LAPACK and special}

\subsection*{Numerical optimization}

\subsection*{Statistical distributions}

\subsection*{Polynomial interpolators}

\subsection*{Test and benchmark suite}


\section*{Project organisation and community}

\textbf{Governance}

\textbf{Roadmap}

\textbf{Community beyond the SciPy library}

\textbf{Maintainers and contributors}


\section*{Discussion}

\textit{The Discussion should be succinct and must not contain subheadings.}

\textbf{Impact now}

\textbf{Future development}.
\textit{This section should include key issues: sparse arrays, ndimage pixel vs point, splines, fftpack vs. np.fft and linalg vs. np.linalg, under-maintained submodules.}


\bibliography{references}

\noindent Use the cite command for an inline citation, e.g. \cite{behnel2011cython}.

\section*{Acknowledgements (not compulsory)}

Acknowledgements should be brief, and should not include thanks to anonymous referees and editors, or effusive comments. Grant or contribution numbers may be acknowledged.

\section*{Author contributions statement}

Must include all authors, identified by initials, for example:
A.A. conceived the experiment(s),  A.A. and B.A. conducted the experiment(s), C.A. and D.A. analysed the results.  All authors reviewed the manuscript.

\section*{Additional information}

To include, in this order: \textbf{Accession codes} (where applicable); \textbf{Competing financial interests} (mandatory statement).

The corresponding author is responsible for submitting a \href{http://www.nature.com/srep/policies/index.html#competing}{competing financial interests statement} on behalf of all authors of the paper. This statement must be included in the submitted article file.

\end{document}
