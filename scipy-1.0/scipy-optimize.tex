\newcommand{\RR}{\ensuremath{\mathbb{R}}}
The \texttt{scipy.optimize} module provides functions for the numerical solution of several classes of root finding and optimization problems:
\begin{enumerate}
\item nonlinear root finding problems (\texttt{brentq}, \texttt{brenth}, \texttt{ridder}, \texttt{bisect}, \texttt{newton}, and \texttt{root}),
\item linear sum assignment problems (\texttt{linear\textunderscore sum\textunderscore assignment}),
\item linear and nonlinear sum-of-squares problems (\texttt{leastsq}, \texttt{least\textunderscore squares}, \texttt{nnls}, \texttt{lsq\textunderscore linear}, and \texttt{curve\textunderscore fit}),
\item general, linear optimization problems (\texttt{linprog}),
\item general, nonlinear, local optimization problems of a single variable (\texttt{minimize\textunderscore scalar}),
\item general, nonlinear, local optimization problems of several variables (\texttt{minimize}), and
\item general, global optimization problems (\texttt{basinhopping}, \texttt{brute}, \texttt{differential\textunderscore evolution}).
\end{enumerate}
Documentation of SciPy's functionality in each these areas can be found in (cite SciPy documentation), but here we highlight recent additions through SciPy 1.0.

%\subsubsection{Root Finding}
%The general ``root finding'' problem is to find a root $\mathbf{x} \in \RR^m$ of $\mathbf{f}: \RR^m \rightarrow \RR^m$, that is, to solve
%\begin{equation}
%\mathbf{f}(\mathbf{x}) = \mathbf{0}
%\end{equation}
%for a solution $\mathbf{x}$.\footnote{Equivalently the problem is to simultaneously find the roots $x_i \in \RR$ of several scalar functions $f_i : \RR \rightarrow \RR$, that is, to solve $f_i(x_0, x_1, \dots, x_{m-1}) = 0$ for $x_i$, $i \in \{0, 1, \dots {m-1}\}$.} The function \texttt{scipy.optimize.root} provides a common interface to several algorithms for solving problems of this type. For the special case\footnote{that is, to solve a single scalar equation $f(x) = 0$ for a single scalar variable $x$} $m = 1$, any one of several specialized functions \texttt{brentq}, \texttt{brenth}, \texttt{ridder}, \texttt{bisect}, or \texttt{newton} may provide improved performance or accuracy. (Have there been any recent improvements? Do we want to summarize the methods as @antonior92 has done for $minimize$? Do we have to explain that these methods only provide \emph{one} solution, and that they are iterative based on a user-provided guess? Do we have to explain the notion of tolerance? Is this a good template for the beginning of the following subsections?)


\subsubsection*{Linear Optimization}

A new interior-point optimizer for continuous linear programming problems, \texttt{linprog} with \texttt{method='interior-point'}, was released with SciPy 1.0. Implementing the core algorithm of the commercial solver MOSEK \cite{andersen2000mosek}, it solves all of the 90+ NETLIB LP benchmark problems \cite{netlib} tested. Unlike some interior point methods, this homogenous self-dual formulation provides certificates of infeasibility or unboundedness as appropriate. 

A presolve routine based on \cite{andersen1995presolving} solves trivial problems and otherwise performs problem simplifications, such as bound tightening and removal of fixed variables, and one of several routines for eliminating redundant equality constraints is automatically chosen to reduce the chance of numerical difficulties caused by singular matrices. Although the main solver implementation is pure Python, end-to-end sparse matrix support and heavy use of SciPy's compiled linear system solvers --- often for the same system with multiple right hand sides due to the predictor-corrector approach --- provide speed sufficient for problems with tens of thousands of variables and constraints.

Compared to the previously implemented simplex method, the new interior-point method is faster for all but the smallest problems, and is suitable for solving medium- and large-sized problems on which the existing simplex implementation fails. However, the interior point method typically returns a solution near the center of an optimal face, yet basic solutions are often preferred for sensitivity analysis and for use in mixed integer programming algorithms. This motivates the need for a crossover routine or a new implementation of the simplex method for sparse problems in a future release, either of which would require an improved sparse linear system solver with efficient support for rank-one updates

\subsubsection*{Nonlinear Optimization}


The \texttt{minimize} function provides methods for solving general nonlinear optimization
problems. It unifies several methods with a common interface in order to facilitate
interchanging between solvers with different characteristics.  Table~\ref{tab:minimize} summarizes
all the methods available for solving minimization problems of the type:
\begin{equation}
  \label{eq:minimization-prob}
  \min_x f(x)
\end{equation}
where $f$ is a multivariate function $f: \mathbb{R}^n \rightarrow \mathbb{R}$ and $x \in \mathbb{R}^n$.
\texttt{COBYLA} and \texttt{SLSQP} methods allow nonlinear constraints to be included in the optimization
problems, \texttt{L-BFGS-B} and \texttt{TNC} allow box constraints $x^l \le x \le x^u$,
and the remaining methods are unconstrained optimization methods.

\begin{table}[]
  \centering
  \caption{Optimization methods from \texttt{minimize}.  The field \textit{version added} gives when the algorithm
    was included in SciPy. Algorithm with \textit{version added} equals to 0.6* were added in version 0.6 or before.
    The field \textit{wrapper} indicates whether the implementation available in SciPy wraps a function written in a compiled language
    (e.g. C or FORTRAN). The fields \textit{\nth{1}} and \textit{\nth{2} derivatives}
    indicates whether the first or second order derivatives are required. When \textit{\nth{2} derivatives} is flagged
    with $\sim$ the algorithm does not requires the second-order derivatives from
    the user, it computes an approximation internally and uses it to accelerate the method convergence.
    \textit{Iterative Hessian factorization} indicates algorithms that factorize the Hessian in an iterative way,
    which does not require the explicit matrix factorization or storage of the Hessian.
    \textit{Local convergence} gives a lower bound on the rate of convergence of the iterations sequence once the
    iterate is sufficiently close to the solution. The convergence rates mentioned at the table
    are: linear (L), superlinear (S) and quadratic (Q). When the convergence rate is denoted by S$^*$ the algorithm
    has a superlinear rate for the parameters used in SciPy, but could  achieve a quadratic convergence rate for other parameter choices.
    The \textit{global convergence} is marked for the algorithms with guarantees of convergence to a stationary
    point (i.e. a point $x^*$ for which $\nabla f(x^*) = 0$). The table also indicates which algorithms
    can deal with constraints on the variables. We distinguish between: \textit{bound constraints} (i.e. $x^l \le x \le x^u$),
    \textit{equality constraints} (i.e. $c_{\text{eq}}(x) = 0$) and \textit{inequality constraints} (i.e. $c_{\text{ineq}}(x) \ge 0$).}
  \begin{tabular}{cccccccccccccc}
      & \rotatebox{80}{\texttt{Nelder-Mead}} & \rotatebox{80}{\texttt{Powell}} & \rotatebox{80}{\texttt{COBYLA}} & \rotatebox{80}{\texttt{CG}} & \rotatebox{80}{\texttt{BFGS}}&  \rotatebox{80}{\texttt{L-BFGS-B}} & \rotatebox{80}{\texttt{SLSQP}} & \rotatebox{80}{\texttt{TNC}} & \rotatebox{80}{\texttt{Newton-CG}} & \rotatebox{80}{\texttt{dogleg}} & \rotatebox{80}{\texttt{trust-ncg}} & \rotatebox{80}{\texttt{trust-exact}} & \rotatebox{80}{\texttt{trust-krylov}} \\
    \hline
    Version added &  0.6* &  0.6* &  0.6* &  0.6* &  0.6* &  0.6* &  0.9 &  0.6* &  0.6* & 0.13 & 0.13 & 0.19 & 1.0 \\
    \hline
    Wrapper & & & \cmark & & & \cmark & \cmark & \cmark & &  & & & \cmark \\
    \hline
    \nth{1} derivatives &  & & & \cmark  & \cmark & \cmark & \cmark & \cmark & \cmark & \cmark & \cmark & \cmark & \cmark \\
    \hline
    \nth{2} derivatives &  &  &  &  & $\sim$ & $\sim$ & $\sim$ & \cmark & \cmark & \cmark & \cmark & \cmark & \cmark \\
    \hline
    \makecell{Iterative Hessian \\
    factorization} & & & &  & & & & \cmark & \cmark &  & \cmark &  & \cmark \\
    \hline
    Local convergence& & & & L & S &  L & S & S$^*$ & S$^*$ & Q & S$^*$ & Q & S$^*$  \\
    \hline
    Global convergence & & &  &   & \cmark & \cmark & \cmark & \cmark & \cmark & \cmark & \cmark & \cmark & \cmark  \\
    \hline
    \makecell{Line-search (LS) or\\ trust-region (TR)} & Neither  & LS &  TR & LS & LS & LS & LS & LS & LS & TR & TR & TR & TR \\
    \hline
    Bound constraints &&&\cmark&&&&\cmark&\cmark&\cmark&&&& \\
    \hline
    Equality constraints &&&&&&&\cmark&&&&& \\
    \hline
    Inequality constraint &&&\cmark&&&&\cmark&&&&& \\
    \hline
    References & \cite{nelder_simplex_1965, wright_direct_1996} & \cite{powell_efficient_1964} &
      \cite{powell_direct_1994, powell_direct_1998, powell_view_2007} &
      \cite{polak_note_1969, nocedal_numerical_2006} & \cite{nocedal_numerical_2006} & \cite{byrd_limited_1995, zhu_algorithm_1997} &
      \cite{schittkowski_nonlinear_1982, schittkowski_nonlinear_1982-1, schittkowski_convergence_1983, kraft_software_1988} &
      \cite{nash_newton-type_1984} & \cite{nocedal_numerical_2006}  & 
      \cite{powell_new_1970, nocedal_numerical_2006} &  \cite{steihaug_conjugate_1983, nocedal_numerical_2006} &
      \cite{conn_trust_2000, more_computing_1983} & \cite{gould_solving_1999, lenders_trlib:_2016} \\
    \hline
  \end{tabular}
  \label{tab:minimize}
\end{table}


The available methods implement a variety of strategies for solving optimization problems.
Derivative-free methods (e.g. \texttt{Nelder-Mead}, \texttt{Powell} and \texttt{COBYLA}) are appropriate for minimizing non-differentiable or
noisy objective functions. When the derivatives are available, however, other methods offer faster convergence rates
and have the convergence to a local solution proved.

Available derivative-based methods start with an initial guess and refine it iteratively, seeking
a local solution to the optimization problem. There are algorithms implementing two competing strategies:
line-search and trust-region. The so-called line-search (LS) methods at each iteration choose a direction
and find, approximately, the optimal point in the given direction. Trust-region (TR) methods, on the other
hand, choose a region for which the local approximation is valid and find, approximately, the  best point
inside this \textit{trust-region}.

An example of trust-region method introduced in the latest releases is \texttt{trust-exact},
which a  provides a fast convergence rate by solving  almost exactly the trust-region subproblem.
Its downside is it requires the Hessian to be stored and factored every iteration, which may preclude
the solution of very large problems.

Some of the available optimization methods are appropriate for large-scale problems
(e.g.~problems with more than 1000 variables).
\texttt{CG} does not use second derivatives and is adequate for large-scale problems due to the low storage requirements.
\texttt{L-BFGS-B} maintain a simple and compact representation of the Hessian and is also adequate for large-scale problems.
Finally, \texttt{Newton-CG}, \texttt{trust-ncg} and \texttt{trust-krylov} use the second order derivatives to achieve faster
convergence but are still appropriate for large problems because they factorize the Hessian in an inexact and iterative way,
which does not require the Hessian to be stored.

The method \texttt{trust-krylov}, introduced in the latest release, is as a good compromise between
\texttt{trust-ncg} and \texttt{trust-exact}: it provides a more accurate solution to the trust-region
subproblem than \texttt{trust-ncg} but does not require the Hessian to be stored and factored as in \texttt{trust-exact}.
