The SciPy library is organized as a collection of subpackages.  The 16
subpackages include mathematical building blocks (e.g. linear algebra, Fourier
transforms, special functions), data structures (e.g. sparse matrices, k-D trees),
algorithms (e.g. numerical optimization and integration, clustering, interpolation,
graph algorithms, computational geometry), and higher-level data analysis
functionality (e.g. signal and image processing, statistical methods).

Here we summarize the scope and capabilities of each subpackage.

\begin{description}

\item[\texttt{cluster}] ~ \newline
    The \texttt{cluster} subpackage contains two submodules:
    \texttt{cluster.vq} provides vector quantization and k-means algorithms,
    and \texttt{cluster.hierarchy} provides functions for hierarchical and
    agglomerative clustering.
\item[\texttt{constants}] ~ \newline
    Physical and mathematical constants, including the CODATA recommended
    values of the fundamental physical constants\cite{CODATA2014}.
\item[\texttt{fftpack}] ~ \newline
    Fast Fourier Transform routines.  In addition to the FFT, the subpackage
    includes functions for the discrete sine and cosine transforms and for
    pseudo-differential operators.
\item[\texttt{integrate}] ~ \newline
    The \texttt{integrate} subpackage provides tools for the numerical
    computation of single and multiple definite integrals, and for the
    solution of ordinary differential equations, including initial value
    problems and two-point boundary value problems.
\item[\texttt{interpolate}] ~ \newline
    The \texttt{interpolate} subpackage contains spline functions and
    classes, one-dimensional and multi-dimensional (univariate and
    multivariate) interpolation classes, Lagrange and Taylor polynomial
    interpolators, and wrappers for FITPACK and DFITPACK functions.
\item[\texttt{io}] ~ \newline
    A collection of functions and classes for reading and writing MATLAB, IDL,
    Matrix Market, Fortran, NetCDF, Harwell-Boeing, WAV and ARFF data files. 
\item[\texttt{linalg}] ~ \newline
    Linear algebra functions, including:
    elementary functions of a matrix, such as the trace, determinant, norm and
    condition number;
    basic solver for $Ax = b$;
    specialized solvers for Toeplitz matrices, circulant matrices, triangular
    matrices and other structured matrices; least squares solver and
    pseudo-inverse calculations; eigenvalue and eigenvector calculations
    (basic and generalized); matrix decompositions, including Cholesky, Schur,
    Hessenberg, $LU$, $LDL^{\intercal}$, $QR$, $QZ$, singular value, and polar;
    and functions to create specialized matrices, such as diagonal, Toeplitz,
    Hankel, companion, Hilbert, and more.
\item[\texttt{ndimage}] ~ \newline
    This package contains various functions for multi-dimensional image
    processing, including: convolution and assorted linear and nonlinear
    filters (Gaussian filter, median filter, Sobel filter, etc.);
    interpolation; region labeling and processing; and mathematical morphology
    functions.
\item[\texttt{misc}] ~ \newline
    A collection of functions that did not fit into the other subpackages.
    While this subpackage still exists in version 1.0.0, effort is continuing
    to deprecate the contents of this subpackage and eventually remove it.
\item[\texttt{odr}] ~ \newline
    Orthogonal distance regression, including Python wrappers for the Fortran
    library ODRPACK.
\item[\texttt{optimize}] ~ \newline
    The package includes the following, with additional details in the SI:
    implementations of many minimization algorithms; a general linear
    programming solver; a routine for least-squares curve fitting; and a
    collection of general nonlinear solvers for root-finding.
\item[\texttt{signal}] ~ \newline
    The \texttt{signal} subpackage focuses on signal processing (plus some
    basic linear systems theory).  Functionality includes:
    convolution and correlation; splines; filtering and filter design;
    continuous and discrete time linear systems; waveform generation;
    window functions; wavelet computations; peak finding; and spectral
    analysis.  
\item[\texttt{sparse}] ~ \newline
    This package includes implementations of several representations of
    sparse matrices.  It contains two subpackages, 
    \texttt{scipy.sparse.linalg} and \texttt{scipy.sparse.csgraph}.

    \texttt{scipy.sparse.linalg} provides a collection of linear algebra
    routines that work with sparse matrices, including linear equation
    solvers, eigenvalue decomposition, singular value decomposition
    and LU factorization.

    \texttt{scipy.sparse.csgraph} provides a collections of graph algorithms
    for which the graph is represented using a sparse matrix.  Algorithms
    include connected components, shortest path, minimum spanning tree
    and more.
\item[\texttt{spatial}] ~ \newline
    This subpackage provides spatial data structures and algorithms,
    including the k-d tree, Delaunay triangulation, convex hulls and Voronoi
    diagrams.  The subpackage \texttt{scipy.spatial.distance} provides
    a large collection of distance functions, along with functions for
    computing the distance between all pairs of vectors in a given collection
    of points or between all pairs from two collections of points.
\item[\texttt{special}] ~ \newline
    The name comes the class of functions traditionally known as \emph{special
    functions}, but over time, the module has grown to include functions
    beyond the classical special functions.  A more appropriate characterization
    of this subpackage is simply \emph{useful functions}.
    It includes: a large collection of the classical special functions
    such as Airy, Bessel, etc.; orthogonal families of polynomials;
    the Gamma function, and functions related to it;
    functions for computing the PDF, CDF and quantile function for several
    probability distributions;
    information theory functions;
    combinatorial functions \texttt{comb} and \texttt{factorial};
    and more.
\item[\texttt{stats}] ~ \newline
    The \texttt{stats} subpackage provides: a large collection of continuous
    and discrete \emph{probability distributions}, each with methods to compute
    the PDF or PMF, CDF, moments and other statistics, generation of random
    variates, and more;
    \emph{statistical tests}, including Pearson's correlation, Spearman's rank-order
    correlations, Kendall's tau, chi-squared test and its generalization as the
    Cressie-Read power divergence, contingency table tests including Fisher's
    exact test and Mood's median test, and many more;
    and assorted \emph{transformations and statistics} of data.
\end{description}
