\documentclass[fleqn,10pt]{wlscirep}
\title{SciPy - some descriptive title here}

\author[1]{Pauli Virtanen}
\author[2,*]{Ralf Gommers}
\author[3]{TODO}
\affil[1]{Affiliation, department, city, postcode, country}
\affil[2]{Affiliation, department, city, postcode, country}
\affil[2]{Affiliation, department, city, postcode, country}

\affil[*]{ralf.gommers@gmail.com}

\keywords{Scientific computing, Python, Mathematics}

\begin{abstract}
TODO. Abstract must be under 200 words and not include subheadings or citations.
\end{abstract}
\begin{document}

\flushbottom
\maketitle
\thispagestyle{empty}

\section*{Introduction}

\textit{The Introduction section expands on the background of the work (some overlap with the Abstract is acceptable). The introduction should not include subheadings.}

\textbf{History}

\textbf{Project goals and scope}

\textbf{Current status (maturity, users)}


\section*{Architecture and implementation choices}

\subsection*{Submodule organisation}

\subsection*{Common infrastructure}

\subsection*{Language preferences}

\subsection*{API and ABI evolution}


\section*{Key technical improvements}

Here we describe key technical improvements made in the last three years.

\subsection*{Data structures}
\textbf{cKDTree}
The \texttt{cKDTree} module, which implements a space-partitioning data structure that
organizes points in $k$-dimensional space, was rewritten in C++ with templated classes. 
Support was added for periodic boundary conditions, which are often used 
in simulations of physical processes. 

In 2013, the time complexity of the $k$-nearest neighbor search from
\texttt{cKDTree.query} was approximately loglinear \cite{knn-jake},
consistent with its formal description \cite{kdtree-search-algo}.
Since then, we've enhanced \texttt{cKDTree.query} by reimplementing it in
C++, removing memory leaks, and allowing release of the global interpreter lock (GIL) so that
multiple threads may be used\cite{gh-4374}. This generally improved
performance on any given problem (Figure~\ref{fig:asvbench}) while
preserving the asymptotic complexity (Figure~\ref{fig:knn-complexity}).

In 2015, SciPy added the \texttt{sparse\_distance\_matrix} routine for generating 
approximate sparse distance matrices between \texttt{KDTree} objects by ignoring 
all distances that exceed a user-provided value. Also, this routine is not 
limited to the conventional L2 (Euclidean) norm but supports any Minkowski 
p-norm between 1 and infinity. By default, the returned data structure is a 
Dictionary Of Keys (DOK) based sparse matrix, which is very efficient for matrix 
construction. This hashing approach to sparse matrix assembly can be 7 times 
faster than constructing with CSR format
\cite{10.1007/978-3-540-75755-9_107}, and the C++ level sparse matrix construction 
releases the Python GIL for increased performance. Once the matrix is constructed, 
distance value retrieval has an amortized constant time complexity 
\cite{Cormen:2001:IA:580470}, and the DOK structure can be efficiently converted 
to a CSR, CSC, or COO matrix to allow for 
speedy arithmetic operations.

In 2015 the \texttt{cKDTree} dual tree counting algorithm\cite{Moore2000ar}
was enhanced to support weights\cite{ckdtree-weights}, which are
essential in many scientific applications, e.g. computing correlation
functions of galaxies\cite{0004-637X-750-1-38}.

\textbf{Sparse matrices}

\subsection*{Unified bindings to compiled code}
LowLevelCallable

\subsection*{Cython bindings for BLAS, LAPACK and special}

\subsection*{Numerical optimization}

\subsection*{Statistical distributions}

\subsection*{Polynomial interpolators}

\subsection*{Test and benchmark suite}


\section*{Project organisation and community}

\textbf{Governance}

\textbf{Roadmap}

\textbf{Community beyond the SciPy library}

\textbf{Maintainers and contributors}


\section*{Discussion}

\textit{The Discussion should be succinct and must not contain subheadings.}

\textbf{Impact now}

\textbf{Future development}.
\textit{This section should include key issues: sparse arrays, ndimage pixel vs point, splines, fftpack vs. np.fft and linalg vs. np.linalg, under-maintained submodules.}


\bibliography{references}

\noindent Use the cite command for an inline citation, e.g. \cite{behnel2011cython}.

\section*{Acknowledgements (not compulsory)}

Acknowledgements should be brief, and should not include thanks to anonymous referees and editors, or effusive comments. Grant or contribution numbers may be acknowledged.

\section*{Author contributions statement}

Must include all authors, identified by initials, for example:
A.A. conceived the experiment(s),  A.A. and B.A. conducted the experiment(s), C.A. and D.A. analysed the results.  All authors reviewed the manuscript.

\section*{Additional information}

To include, in this order: \textbf{Accession codes} (where applicable); \textbf{Competing financial interests} (mandatory statement).

The corresponding author is responsible for submitting a \href{http://www.nature.com/srep/policies/index.html#competing}{competing financial interests statement} on behalf of all authors of the paper. This statement must be included in the submitted article file.

\end{document}
