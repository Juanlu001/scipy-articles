The cKDTree module was rewritten in C++ with templated classes, and support for
periodic boundary conditions was added. A periodic boundary condition is typically 
used in computer simulations of physical processes.

% TODO: please check / improve literature citations used here
% what about the time complexity of the kNN implementation?
In 2015, we enhanced \texttt{cKDTree.query} with a $k$ nearest neighbors search
parameter. This is an efficient operation\cite{Sproull:1991:RNS:3118219.3118331} 
because it generates and returns a data structure that may be only a fraction 
of the size of the full neighbor list. A C++ implementation is provided that releases
the Python global interpreter lock for the neighbor search process, uses
a memory pool to allocate and automatically reclaim \texttt{structs}, and
heapsort to sort the $k$ nearest neighbors distances, even in the case
of periodic boundary conditions. When $k$ is a non-contiguous list of nearest
neighbors to account for (i.e., \texttt{[1, 3, 4]}), the intervening distances
between the nearest neighbor ($k = 1$) and the farthest neighbor requested
are all interrogated, but only those neighbors that are requested are retained
in memory. Thus, the maximum heap size for this algorithm is conceptually determined
as \texttt{np.arange(max(k))}.

In 2015, SciPy added support for generating approximated sparse distance matrices 
between \texttt{KDTree} objects by zeroing all distances that exceed a user-provided value. 
The distance metric used between k-d trees is not constrained to the 
conventional L2 (Euclidean) norm---any Minkowski p-norm value bounded by 1 and 
infinity is valid. By default, the returned data structure is a dictionary of 
keys based sparse matrix, which is very efficient for matrix construction. 
Previously-published results had suggested that this hashing approach to sparse 
matrix assembly is 7 times faster than constructing with compressed row format (CSR)
\cite{10.1007/978-3-540-75755-9_107} and the C++ level sparse matrix construction 
releases the Python GIL for increased performance. Also, once the matrix is 
constructed, distance value retrieval has an amortized constant time complexity 
\cite{Cormen:2001:IA:580470}. For efficiently performing arithmetic operations 
on it, SciPy allows the sparse matrix to be directly converted to the common 
CSR, CSC and COO data structures.

The cKDTree module implements a dual tree counting algorithm\cite{Moore2000ar},
with an improvement to the pair counting algorithm to improve the scaling
with the number of bins. The cKDTree can now be augmented by weights, with 
weighted paircount essential in many scientific applications, e.g. computing 
correlation functions of galaxies\cite{0004-637X-750-1-38}.

\fixme{Add a Figure to show the scaling, before and after.}
\fixme{perhaps give an example or some formula.}
(cite / mention faster implementions of paircounting algorithms / treecorr, corrfunc)

The main purpose of the paircounting algorithm in cKDTree is to provide a readily
available tool. The KDTree based algorithm is not necessarily the fastest algorithm
depending on the application. E.g. for low number densities a chaining mesh based algorithm
can be easily many times faster than a tree based algorithm.

